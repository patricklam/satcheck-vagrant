%-----------------------------------------------------------------------------
%
%               Template for sigplanconf LaTeX Class
%
% Name:         sigplanconf-template.tex
%
% Purpose:      A template for sigplanconf.cls, which is a LaTeX 2e class
%               file for SIGPLAN conference proceedings.
%
% Guide:        Refer to "Author's Guide to the ACM SIGPLAN Class,"
%               sigplanconf-guide.pdf
%
% Author:       Paul C. Anagnostopoulos
%               Windfall Software
%               978 371-2316
%               paul@windfall.com
%
% Created:      15 February 2005
%
%-----------------------------------------------------------------------------


\documentclass[10pt,nocopyrightspace,preprint]{sigplanconf}

%\usepackage{SIunits}            % typset units correctly
%\usepackage{courier}            % standard fixed width font
\usepackage{times}
\usepackage[scaled]{helvet} % see www.ctan.org/get/macros/latex/required/psnfss/psnfss2e.pdf
\usepackage{url}                  % format URLs
\usepackage{listings}          % format code
\usepackage{enumitem}      % adjust spacing in enums
\sloppy
\usepackage{booktabs}
\usepackage[colorlinks=true,allcolors=blue,breaklinks,draft=false]{hyperref}
% hyperlinks, including DOIs and URLs in bibliography
% known bug: http://tex.stackexchange.com/questions/1522/pdfendlink-ended-up-in-different-nesting-level-than-pdfstartlink
\newcommand{\doi}[1]{doi:~\href{http://dx.doi.org/#1}{\Hurl{#1}}}   % print a hyperlinked DOI


% The following \documentclass options may be useful:
%
% 10pt          To set in 10-point type instead of 9-point.
% 11pt          To set in 11-point type instead of 9-point.
% authoryear    To obtain author/year citation style instead of numeric.

\usepackage{amsmath, amsthm, amssymb}
\usepackage{graphicx}
\usepackage{xspace}

\newtheorem{thm}{Theorem}[section]


\usepackage{lmodern}
\usepackage{color}
\usepackage{textcomp}
\definecolor{listinggray}{gray}{0.9}
\definecolor{lbcolor}{rgb}{0.9,0.9,0.9}
\lstset{
	% backgroundcolor=\color{lbcolor},
	% rulecolor=,
	% aboveskip={1.5\baselineskip},
	% extendedchars=true,
	tabsize=3,
	language=C++,
	basicstyle=\ttfamily\scriptsize,
	upquote=true,
	columns=fixed,
	showstringspaces=false,
	breaklines=true,
	prebreak = \raisebox{0ex}[0ex][0ex]{\ensuremath{\hookleftarrow}},
	% frame=single,
	showtabs=false,
	showspaces=false,
	showstringspaces=false,
	identifierstyle=\ttfamily,
	keywordstyle=\color[rgb]{0,0,1},
	commentstyle=\color[rgb]{0.133,0.545,0.133},
	stringstyle=\color[rgb]{0.627,0.126,0.941},
	numbers=left,
	numberstyle=\tiny,
	numbersep=8pt,
	escapeinside={/*@}{@*/}
}

\newcommand{\SC}[0]{\ensuremath{\xrightarrow{\textit{sc}}}\xspace}
\newcommand{\rf}[0]{\ensuremath{\xrightarrow{\textit{rf}}}\xspace}

%% Not used yet
% \usepackage{times}
% \usepackage{latexsym}
% \usepackage{amsfonts}
% \usepackage{amsthm}
% \usepackage{txfonts}
% \usepackage{url}
% \usepackage{subfigure}

% \usepackage{bbm}
% \usepackage{stfloats}

\newcommand{\code}[1]{\text{\tt #1}}
\newcommand{\codebf}[1]{\text{\tt\bfseries #1}}

\newcommand{\TODO}[0]{\textbf{TODO}\xspace}
\newcommand{\todo}[1]{{\bf [[#1]]}}
%\newcommand{\todo}[1]{}
\newcommand{\comment}[1]{}
\newcommand{\st}[0]{st\xspace}
\newcommand{\expr}[0]{expr\xspace}
\newcommand{\naive}[0]{na\"{i}ve\xspace}
\newcommand{\Naive}[0]{Na\"{i}ve\xspace}
\newcommand{\naively}[0]{na\"{i}vely\xspace}
\newcommand{\Naively}[0]{Na\"{i}vely\xspace}
\newcommand{\tool}[0]{SATCheck\xspace}
\newcommand{\Tool}[0]{SATCheck\xspace}
\newcommand{\tuple}[1]{\ensuremath \langle #1 \rangle}
\newcommand{\size}[1]{\ensuremath \left| #1 \right|}
\newcommand{\roundup}[1]{\ensuremath \lceil #1 \rceil}

\begin{document}

\conferenceinfo{OOPSLA '15}{date, City.} 
\copyrightyear{2015} 
\copyrightdata{[to be supplied]} 

\sloppy

\title{Step-by-Step Guide for SATCheck Artifact}
\authorinfo{Patrick Lam and Brian Demsky}{}{}

\maketitle

The Getting Started guide explains how to reproduce the results in the
paper, up to the generation of the graphs. This document provides more
information about the parts of the artifact and how they specifically
connect to the paper.

\section{Example}
The motivating example corresponds to the {\tt linuxlock} example with
problem size 1. Recall that you use {\tt bench.sh} to run a benchmark;
to only run problem size 1, change {\tt SIZES} in the {\tt benchmark-config.sh} file
in the benchmark directory.

We'll discuss each part of the example below.

The example discusses the SATCheck main loop. The code for the main loop is in
{\tt model.cc}. Execution points are implemented in the {\tt
  ExecPoint} class, found in {\tt execpoint.cc}.

\section{Event Graph \& Encoding}
If you add {\tt DUMP\_EVENT\_GRAPHS} to {\tt config.h}, you will get
the event graphs in {\tt .dot} format when you run a program under
SATCheck. The graphs in the paper have been trimmed and we've manually
extracted subgraphs to better illustrate our points.

The event graph is stored in memory as a tree
of {\tt EPRecord} objects, manipulated by the {\tt ConstGen} class. 
We implemented the value set analysis and encodings in the {\tt
  StoreLoadSet} class.

Section 4 in the paper is a fairly literal description of the {\tt
  ConstGen} class, and we don't think that much needs to be said here.
Note that remaining goals are stored in the {\tt goalset} field of
{\tt ConstGen}. These goals search for unobserved behaviours.

\section{Exploring}
The schedule builder is found in the {\tt ScheduleBuilder} class. That
class includes the construction of the wait pairs, which modifies the
execution's schedule for the next concrete execution to be visited.

\section{Extensions}
\paragraph{Field support}
Most of the work to support fields happens in the front-end; see
Section~\ref{sec:frontend} in this document for information on the
front-end. Our handlers for reads and writes ({\tt LoadHandler} and
{\tt StoreHandler} respectively) account for field accesses and
generate the appropriate instrumentation, which the backend handles.

\paragraph{Sharing between Instances of Uninterpreted Functions.}
Internally, our instrumentation identifies the uninterpreted function
instance using the {\tt MC2\_function\_id()} call. The frontend
assigns a unique function identifier for each static point in the code
that generates an uninterpreted function annotation. In the backend,
uninterpreted functions are handled in {\tt MCExecution::function.}

\paragraph{Incremental Solving.} 
The SAT encoding reuse code can be found in {\tt ConstGen::canReuseEncoding()}.

\paragraph{TSO Extension.} As described in the Getting Started guide, enable
TSO by defining it in {\tt config.h}. The Makefile creates two copies of the runtime library:
one for SC and one for TSO. The implementation of TSO is scattered around
the codebase, but {\tt constgen.cc} generates key TSO constraints.

\section{Test Schedule Generation}
Enabling the {\tt SUPPORT\_MOD\_ORDER\_DUMP} option in {\tt config.h} will
expose schedule graphs in {\tt \.dot} format; see {\tt MCExecution::dumpExecution} 
in {\tt model.cc}. Generating test case specifications would be analogous to
{\tt dumpExecution}'s implementation.

\section{Instrumentation}
\label{sec:frontend}
The Clang front-end lives in the {\tt clang} subdirectory. It takes an
unannotated C file and outputs annotated C code to standard output (as
invoked by {\tt bench.sh}).  It assumes that the C file performs
operations on shared memory using calls to {\tt load\_NN}, {\tt
  store\_NN}, and {\tt rmw\_NN}; the primitives are defined in {\tt
  include/libinterface.h}.

When the AST traversal encounters a shared memory access, it records the
call and inserts instrumentation; see, for example, the {\tt
  LoadHandler} code in {\tt clang/src/add\_mc2\_annotations.cpp}. When
the frontend inserts instrumentation, it also records the variables
being instrumented and, in a second pass, adds additional
instrumentation to indicate uninterpreted functions to the model
checker.

\paragraph{Caveats.} (1) We assume that programs terminate under unfair schedules. 
If you have a spin loop for which this is not true, you need to
manually insert {\tt MC2\_yield()} calls where control flows back to
the top of a loop.  (2) Our front-end also does not get correct
information about source locations for macros.  As a result, it will
not annotate uses of variables of type {\tt bool} (which is
implemented as a macro).

\section{Evaluation}
The Getting Started guide explained how to reproduce our evaluation.
One note: the CAS spinlock example in the paper corresponds to the
{\tt linuxlock} example in the codebase.

The benchmark runs work as follows. The {\tt bench.sh} script is shared among
scripts; it relies on {\tt benchmark-config.sh} to specify behaviour. 
If front-end compilation is appropriate, it runs the Clang frontend on the
unannotated file to generate an annotated file. Then the script iterates on the
requested sizes ({\tt \$SIZES} for non-TSO, {\tt \$SIZES\_TSO} for TSO),
compiles the benchmark with the command-line flag {\tt -DPROBLEMSIZE}, and records
the time to the {\tt log} file and the output to the {\tt logall} file. We then created
a quick and dirty Java parser to read the {\tt log} files as generated by
{\tt bench.sh} and create gnuplot files.

The expectations for each benchmark are 1) that its main function is
called {\tt user\_main()}, and 2) that it uses the shared memory
access functions which we define in {\tt libinterface.h}. Apart from
that, benchmarks should use the standard C11 thread library.  Our {\tt
  bench.sh} script then compiles the benchmark against our {\tt
  libmodel.so} shared library, which contains its own {\tt main()}
function to explore the benchmark's behaviours as described as ``the SATCheck main loop'' in the paper.

We've included scripts for CDSChecker, Nidhugg, and CheckFence, but we did 
not package them as part of our evalution.

\end{document}
